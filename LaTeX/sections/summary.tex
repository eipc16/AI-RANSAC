\documentclass[../main.tex]{subfiles}

\begin{document}
    \paragraph{}
    Wykrywanie obiektów na obrazach jest w pewnym stopniu skomplikowanym procesem, składającym się z wielu etapów. Po ekstrakcji punktów kluczowych charakteryzujących każdy z obrazów, należy na podstawie ich podobieństwa (cech) połączyć je w odpowiednie pary. To właśnie ten proces zajmuje najwięcej czasu, ponieważ znalezienie odpowiednich par ma bardzo dużą złożoność obliczeniową. Przypuścmy, że obraz pierwszy posiada 6000 punktów kluczowych, natomiast obraz drugi 5000. W celu wyznaczenia par punktów kluczowych należy dla każdego punktu z obrazu pierwszego obliczyć podobieństwo do każdego z punktów na obrazie numer dwa i odwrotnie. Następnie dla każdego punktu należy sprawdzić, który z punktów z drugiego obrazu jest do niego najbardziej podobny. Ostatecznie nawet dla niewielkich obrazów musi sprawdzić 30000000 przypadków. Dla większych obrazów liczba ta może wzrosnąć nawet kilkadziesiąt czy kilkaset razy.
    Właśnie z tego powodu w mojej implementacji wykorzystałem moduł `multiprocessing` dostępny w środowisku Python w celu prowadzenia równoległych obliczeń na wielu rdzeniach. Niezwykle przydatny okazał się też moduł `NumPy` pozwalający na przeprowadzanie obliczeń na macierzach i wektorach z niewiarygodną szybkością. 
    
    \paragraph{}
    Usunięcie szumów poprzez badanie spójności sąsiedztwa punktów kluczowych znacznie zmniejsza liczbę branych pod uwagę par. Poprzez wybieranie tych par, które łączą ze sobą zbliżone regiony można usunąć szumy łączące losowe regiony zdjęć. Niezwykle ważny jest jednak odpowiedni dobór parametrów. Zbyt duża liczebność sąsiedztwa lub za duży próg spójności mogą spowodować odfiltrowanie poprawnych par punktów. W przypadku małego sądziedztwa natomiast, lub bardzo niskiego progu spójności, ryzykujemy umieszczenie dużej ilości szumu w zbiorze wynikowym. Warto wspomnieć, że liczenie odległości między punktami można łatwo przekształcić na operacje na macierzach, co po wykorzytaniu modułu 'NumPy' może znacznie zwiększyć wydajność programu.
    
    \paragraph{}
    W kolejnym etapie spójne pary punktów kluczowych wykorzystywane są do znalezienia odpowiedniego model transformacji obrazu, który pozwoli na pozbycie się błędów wynikających z obrotu czy różnych pozycji obiektu na obrazach. Transformaty są obliczane na podstawie losowej próbki danych ze zbioru par punktów kluczowych. W klasycznej wersji metody RANSAC próbka ta jest wybierana całowicie losowo, jednak jak się okazało w badaniach, tworzenie model na podstawie określonych punktów prowadzi do lepszych wyników. Trzeba jednak pamiętać, że metoda estymacji RANSAC nie jest deterministyczna, a wyliczenie odpowiednich parametrów modelu może w niektórych wypadkach zająć dużo czasu. Warto więc ją usprawnić poprzez implementację metod pozwalających na przykładowo estymację liczby wymaganych do uzyskania odpowiedniego modelu iteracji. Podsumowując, RANSAC jest świetnym narzędziem do estymacji parametrów modeli na podstawie zbiorów zawierających dużą liczbę danych odstających, w celu uzyskania lepszych rezultatów jednak warto się także przyjrzeć jego rozszerzeniom MSAC oraz MLESAC.
\end{document}
