\documentclass[../main.tex]{subfiles}

\begin{document}
   Metoda RANSAC jako parametry przyjmuje dwie wartości, liczbę iteracji oraz maksymalną wartość błędu, której przekroczenie powoduje odfiltrowanie pary ze zbioru wynikowego. Do obliczeń wykorzystano transformatę afiniczą z heurystyką wybierającą trzy losowe pozycje ze zbioru par punktów kluczowych.
    
    \begin{table}[H]
    \caption{Wyniki pomiarów czasu w zależności od liczby iteracji}
     \label{t:iterations}
     \begin{center}
        \begin{tabular}{|l|r|r|r|r|}
            \hline
            \multirow{2}{*}{{}} &
            \multicolumn{4}{c|}{\textbf{Liczba iteracji}} \\
            
            \cline{2-5}
            {Przedmiot} & \textbf{10} & \textbf{500} & \textbf{1000} & \textbf{5000} \\
            \cline{2-5}
            {} & \multicolumn{4}{c|}{\textbf{Czas [s]}} \\
            
             \hline
             {Myszka} & {0,04} & {2,70} & {4,57} & {29,97}  \\
             \hline 
             {Kaktus} & {0,05} & {2,41} & {4,82} & {32,48} \\
             \hline
             {Portfel} & {0,12} & {3,31} & {6,84} & {41,89}  \\
            \hline
             {Kubek} & {0,15} & {9,36} & {19,01} & {102,29} \\
             \hline
             {Książka} & {0,21} & {10,84} & {26,92} & {124,54}  \\
            \hline
            
        \end{tabular}
     \end{center}
    \end{table}

    \begin{table}[H]
    \caption{Wyniki działania algorytmu w zależności o wartości błędu maksymalnego}
     \label{t:errs}
     \begin{center}
        \begin{tabular}{|l|r|r|r|r|}
            \hline
            \multirow{2}{*}{{}} &
            \multicolumn{4}{c|}{\textbf{Wartość maksymalnego błędu}} \\
            
            \cline{2-5}
            {Przedmiot} & \textbf{1} & \textbf{20} & \textbf{40} & \textbf{200} \\
            \cline{2-5}
            {} & \multicolumn{4}{c|}{\textbf{Największy konsensus}} \\
            
             \hline
             {Kubek} & {80} & {407} & {407} & {445} \\
             \hline
             {Portfel} & {59} & {222} & {225} & {242}  \\
             \hline
             {Myszka} & {6} & {37} & {40} & {52}  \\
             \hline
             {Książka} & {60} & {645} & {645} & {650}  \\
             \hline
             {Kaktus} & {11} & {12} & {12} & {12} \\
            \hline
        \end{tabular}
     \end{center}
    \end{table}
   
   \paragraph{}
   Jak można zauważyć w badaniach liczba iteracji, zaraz obok rozmiaru list par punktów spójnych jest jednym z najważniejszych czynników wpływających na czas przetwarzania. Przy zbyt małej liczbie iteracji algorytm nie jest w stanie znaleźć optymalnego rozwiązania duża liczba iteracji jednak znacząco wpływa na czas działania programu, podczas gdy rozwiązanie mogło zostać znalezione już na samym początku.
   
   \paragraph{}
   Maksymalny błąd natomiast jest jednym z czynników definiujących skuteczność metody RANSAC, zbyt mały błąd powoduje odfiltrowanie znacznej liczby właściwych par, natomiast zbyt duży pozostawia w danych wiele szumów. Przeprowadzone badania wskazują na to, że dla badanych par obrazków, najbardziej optymalny wartość błędu wynosi od 20 do 40.
   
\end{document}
