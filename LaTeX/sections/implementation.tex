\documentclass[../main.tex]{subfiles}

\begin{document}
    Program składa się z kilku modułów odpowiadających za wyznaczanie par punktów kluczowych, wyszukiwanie par spójnych, wyznaczanie transformat, urchomienie algorytmu RANSAC oraz usprawnianie procesu odczytu i zapisu obrazów oraz plików .json i .harriff.sift. Po podaniu dwóch plików źródłowych (w formacie .png) program korzystając z przekazanego nam skryptu wyznacza punkty kluczowe (regiony kluczowe) na obu obrazach. Następnie rozpoczynany jest proces łączenia punktów w pary na podstawie ich podobieństwa wizualnego (wartości features). W kolejnym kroku zbiór par punktów kluczowych jest filtrowany. W dalszych punktach brane pod uwagę są tylko te punkty, które spełniają kryterium spójności. Pary spójnych punktów kluczowych są następnie przekazywane do następnej części programu, która uruchamia metodę RANSAC, w każdym kroku algorytmu na podstawie próbki danyh wyznaczany jest model. Otrzymany model jest następnie ewaluowany. Oceną modelu jest liczba par punktów kluczowych, których odległość między punktem z obrazu drugiego, a transformowanym punktem z obrazu pierwszego jest mniejsza niż zadana wartość dozwolonego błędu. Na sam koniec najlepszy model jest używany do odfiltrowania par punktów kluczowych, które nie spełniają podanego kryteriu (odległość między punktami jest większa od dozwolonego błędu).
\end{document}
