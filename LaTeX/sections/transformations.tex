\documentclass[../main.tex]{subfiles}

\begin{document}
   \paragraph{}
   W przypadku kiedy obiekt na zdjęciach znajduje się w różnych ustawieniach, czyli np. jest obrócony lub leży pod kątem, podobnieństwo może nie zostać stwierdzone. Rozwiązaniem tego problemu są transformaty, które odpowiadają za przekształcenie jednego obrazu w drugi. Badanie skupia się na dwóch transformatach: afinicznej i perspektywicznej. Transformata afiniczna odpowiada za przekształcenia typu przesunięcie, skalowanie i obrót. Transformata perspektywiczna natomiast rozszerza ją i bierze pod uwagę przekształcenie polegające na spłaszczaniu i rozciąganiu obrazu. Dzięki nim jesteśmy w stanie porównać obiekty nawet w przypadku kiedy oba obrazy pokazują je pod innym kątem.
    
    \paragraph{}
    Badania przeprowadzone zostały dla liczby iteracji równej 5000, dodatkowo próg błędu wynosi 25. Przedstawione rezultaty są średnimi wynikami z pięciu uruchomień. Kolumny Para i Iteracja oznaczają kolejno liczbę otrzymanych w wyniku działania metody RANSAC par oraz, w której iteracji zostało znalezione to rozwiązanie. Czas przetwarzania natomiast oznacza liczbę sekund potrzebną na ukończenie wszystkich pięciu tysięcy iteracji.
    
    \centering
    \begin{table}[H]
    \caption{Porównanie wyników dla transformaty afinicznej i perspektywicznej}
     \label{t:transform}
     \begin{center}
        \begin{tabular}{|l|c|r|p{2.5cm}||r|r|p{2.5cm}|}
            \hline
            \multirow{2}{*}{\textbf{Przedmiot}} &
            \multicolumn{3}{c||}{\textbf{Transformata afiniczna}} &
            \multicolumn{3}{c|}{\textbf{Transformata perspektywiczna}} \\
            
            \cline{2-7} &
            \textbf{Pary} & \textbf{Iteracja} & \textbf{Czas przetwarzania [s]} & 
            \textbf{Pary} & \textbf{Iteracja} & \textbf{Czas przetwarzania [s]} \\

             \hline
             {Kubek} & {406} & {12} & \makecell{}{66,57} & {406} & {14} & \makecell{}{99,43} \\
             \hline
             {Portfel} & {214} & {231} & \makecell{}{31,11} & {239} & {760} & \makecell{}{75,09} \\
             \hline
             {Myszka} & {36} & {1131} & \makecell{}{18,47} & {49} & {638} & \makecell{}{33,42} \\
             \hline
             {Książka} & {595} & {4000} & \makecell{}{90,81} & {648} & {102} & \makecell{}{127,67}  \\
             \hline 
             {Kaktus} & {13} & {3573} & \makecell{}{29,52} & {15} & {3582} & \makecell{}{49,89} \\
            \hline
        \end{tabular}
     \end{center}

    \end{table}
    
        \begin{figure}[H]
        \caption{Wpływ wyboru transformaty na czas przetwarzania}
        \centering
        \begin{tikzpicture}
        \begin{axis}[
            width=10cm,
            xticklabels={'', '', Myszka, Kaktus, Portfel, Kubek, Książka},
            enlargelimits=0.15,
            ylabel={Czas przetwarzania [s]},
            xlabel={Przedmiot},
            ybar
        ]
            \addplot plot coordinates {(1,18.47) (2,29.52) (3,31.11) (4,66.57) (5,90.81)};
            \addplot plot coordinates {(1,33.42) (2,49.89) (3,75.09) (4,99.43) (5,127.67)};
        \end{axis}
        \end{tikzpicture}
        \end{figure}
    
    \justify
    Głównym wnioskiem wysuwającym się po przeprowadzeniu badań jest to, że czas przetwarzania w przypadku transformaty perspektywicznej jest dłuższy. Spowodowane jest to tym, że uwzględnienie transformacji rozciągania i zwężania obrazu wymaga większej liczby obliczeń. W większości przypadków transformata perspektywiczna osiągała lepsze wyniki, ponieważ większość par zdjęć źródłowych różniła się kątem nachylenia przedmiotu. Można także zauważyć, że transformata prespektywicza zwykle znajdowała lepsze wyniki już w początkowych iteracjach. Jednak mogła na to mieć także wpływ losowość metody RANSAC.
    
\end{document}
