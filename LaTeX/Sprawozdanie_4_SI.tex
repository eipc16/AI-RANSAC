\documentclass[a4paper,10pt]{article}
%\documentclass[a4paper,10pt]{scrartcl}

\usepackage{titlesec}
\usepackage{polski}
\usepackage[utf8]{inputenc}
\usepackage[document]{ragged2e}
\usepackage{geometry}
\usepackage{listings}
\usepackage{makecell}
\usepackage{float}
\usepackage{siunitx}
\usepackage{pgfplotstable}
\usepackage{subfiles}
\usepackage{graphicx}

\graphicspath{ {images/} }

\pgfplotsset{compat=newest}
\usepgfplotslibrary{units}

\pdfinfo{%
  /Title    (Sprawozdanie nr 4 Rozpoznawanie obrazów)
  /Author   (Przemysław Pietrzak)
  /Creator  (Przemyslaw Pietrzak)
  /Producer (Przemysław Pietrzak)
  /Subject  (Sztuczna inteligencja)
  /Keywords (ransac, sift, sztuczna, inteligencja)
}

\lstset{frame=tb,
  language=Java,
  aboveskip=3mm,
  belowskip=3mm,
  showstringspaces=false,
  columns=flexible,
  basicstyle={\small\ttfamily},
  numbers=none,
  breaklines=true,
  breakatwhitespace=true,
  tabsize=3
}

\sisetup{
  round-mode          = places,
  round-precision     = 2,
}

\begin{document}
    \begin{titlepage}
     \newgeometry{centering, margin=1.5cm}
     \vspace*{\fill}
    
     \vspace*{-4cm}
     \Huge\bfseries\
     {Sztuczna inteligencja i inżynieria wiedzy}
    
     \LARGE
     \centering
     \vspace{2cm}
     {Sprawozdanie nr 4}
    
     \Large
     \centering
     {Rozpoznawanie obrazów}
     
     \vspace*{0.5cm}
     
     \centering
     \large 
     \vspace{0.5cm}
     Przemysław Pietrzak, 238083
     
     Środa, 17:05
     
     \vspace*{\fill}
     \restoregeometry
    \end{titlepage}
    
    \newpage
    \tableofcontents
    
    \newpage
    \justify
    \section{Opis implementacji}
    \subfile{sections/implementation}
    
    \justify
    \section{Prezentacja działania algorytmu}
    \subfile{sections/run_results}
    
    \newpage
    \justify
    \section{Badanie wpływu parametrów na algorytm spójności sąsiedztwa}
    test
    
    \subsection{Liczba sąsiadów}
    xx
    
    \subsection{Próg spójności}
    ff
    
    \subsection{Omówienie wyników}
    ff
    
    \newpage
    \justify
    \section{Badanie wpływu użytej transformaty na wyniki metody RANSAC}
    test
    
    \subsection{Transformata afiniczna}
    ff
    
    \subsection{Transformata perspektywiczna}
    ff
    
    \subsection{Omówienie wyników}
    ff
    
    \newpage
    \justify
    \section{Badanie wpływu parametrów  na wyniki metody RANSAC}
    test
    
    \subsection{Liczba iteracji}
    ff
    
    \subsection{Maksymalny błąd}
    ff
    
    \subsection{Omówienie wyników}
    ff
    
    \newpage
    \justify
    \section{Badanie wpływu heurystyk na wyniki metody RANSAC}
    test
    
    \subsection{Random}
    ff
    
    \subsection{Heurystyka  1}
    ff
    
    \subsection{Heurystyka 2}
    ff
    
    \subsection{Heurystyka 3}
    ff
    
    \subsection{Heurystyka 4}
    ff
    
    \subsection{Heurystyka 5}
    ff
    
    \subsection{Omówienie wyników}
    ff
    
    \newpage
    \justify
    \section{Podsumowanie}
    xxc
    
\end{document}
